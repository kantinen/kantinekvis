\documentclass[a4paper]{article}

\usepackage[a4paper,bottom=1cm,top=2cm]{geometry}
\usepackage[utf8]{inputenc}
\usepackage[T1]{fontenc}
\usepackage[danish]{babel}
\usepackage{microtype}
\usepackage{palatino}
\pagestyle{empty}

\begin{document}

\begin{center}
{\Huge Kantinequizzen} \\
\vspace{0.4cm}

\begin{tabular}{|p{12cm}|p{0.2cm}|p{0.2cm}|p{0.2cm}|}
\hline
\vspace{0.2cm}
& \texttt{0} & \texttt{X} & \texttt{1} \\\hline
\parbox{12cm}{
  \vspace{0.2cm}
  \textbf{0) Hvad hedder skabet ved komfurerne, hvor man kan finde
  efterladte madvarer til fri afbenyttelse?} \\
  \texttt{0}: Free-food-basaren \quad
  \texttt{X}: Hjørneskabet \quad
  \texttt{1}: Gratishylden
  \vspace{0.1cm}
} & & & \\\hline
\parbox{12cm}{
  \vspace{0.2cm}
  \textbf{1) Hvor findes glasskraldespanden der er forbeholdt glassaffald
  uden pant?} \\
  \texttt{0}: Ved vinduet ved automaterne. \quad
  \texttt{X}: I de store containere udenfor UP1. \\
  \texttt{1}: Overfor KEN.
  \vspace{0.1cm}
} & & & \\\hline
\parbox{12cm}{
  \vspace{0.2cm}
  \textbf{2) Hvilken slags maskine er KEN?} \\
  \texttt{0}: En opvaskemaskine \quad
  \texttt{X}: En kaffemaskine \quad
  \texttt{1}: En desinficeringsmaskine
  \vspace{0.1cm}
} & & & \\\hline
\parbox{12cm}{
  \vspace{0.2cm}
  \textbf{3) Hvor gammel er Kantinen?}\\
  \texttt{0}: 1337 \quad
  \texttt{X}: 44 \quad
  \texttt{1}: 20
  \vspace{0.1cm}
} & & & \\\hline
\parbox{12cm}{
  \vspace{0.2cm}
  \textbf{4) Hvilket socialt arrangement afholder Kantinebestyrelsen
  \emph{ikke}?} \\
  \texttt{0}: Klippeklistredag. \quad
  \texttt{X}: Pigemiddag \quad
  \texttt{1}: Julefrokost.
  \vspace{0.1cm}
} & & & \\\hline
\parbox{12cm}{
  \vspace{0.2cm}
  \textbf{5) Hvad er den vigtigste regel i kantinen?} \\
  \texttt{0}: Drik to liter sort kaffe hver DAAAAAAAAAAAG. \quad\\
  \texttt{X}: Ryd op efter dig selv! \quad
  \texttt{1}: Den vigtiste regel er at der ikke er nogle regler.
  \vspace{0.1cm}
} & & & \\\hline
\parbox{12cm}{
  \vspace{0.2cm}
  \textbf{6) Hvem er Kantinebestyrelsen (vælg kun én)?} \\
  \texttt{0}: En flok kapitalistiske pengepugere. \quad
  \texttt{X}: Sultne, frivillige medstuderende. \\
  \texttt{1}: Rengøringspersonale i forklædning.
  \vspace{0.1cm}
} & & & \\\hline
\parbox{12cm}{
  \vspace{0.2cm}
  \textbf{7) Hvor kan du finde kaffe og te, når der er løbet tør?} \\
  \texttt{0}: I skabet ovenover krus og glas. \quad
  \texttt{X}: Inde hos algoritmikerne på første sal. \\
  \texttt{1}: Trækker det i den rødeste automat.
  \vspace{0.1cm}
} & & & \\\hline
\parbox{12cm}{
  \vspace{0.2cm}
  \textbf{8) Hvornår er det \emph{ikke} tilladt at se YouTube-videoer for fuld
  udblæsning?} \\
  \texttt{0}: Når kantinen er fyldt med flittige studerende. \quad
  \texttt{X}: Til din kandidatfest. \\
  \texttt{1}: Under eksamen (eller reeksamen).
  \vspace{0.1cm}
} & & & \\\hline
\parbox{12cm}{
  \vspace{0.2cm}
  \textbf{9) Hvordan får jeg lettest fat i Kantinebestyrelsen?} \\
  \texttt{0}: Spiller Windows-lyde rigtig højt mens du venter \quad \\
  \texttt{X}: Tager fat i os fysisk eller skriver til \texttt{bestyrelsen@kantinen.org}. \\
  \texttt{1}: Ringer til vores forældre
  \vspace{0.1cm}
} & & & \\\hline
\parbox{12cm}{
  \vspace{0.2cm}
  \textbf{10) Hvor skal brugte pizzabakker smides ud?} \\
  \texttt{0}: I dåsepant-spanden. \quad
  \texttt{X}: I papburet. \quad
  \texttt{1}: I de sorte affaldssække.
  \vspace{0.1cm}
} & & & \\\hline
\parbox{12cm}{
  \vspace{0.2cm}
  \textbf{11) Min mad i brugerkøleskabet er forsvundet. Hvorfor mon?}\\
  \texttt{0}: Fordi der ikke stod navn og dato på \quad \\
  \texttt{X}: Du blev udvalgt i den ugentlige mad-decimering (\textit{lucky you,} altså!) \quad \\
  \texttt{1}: Men, men men .. din mad har jo været død \textit{i tretten år..!}
  \vspace{0.1cm}
} & & & \\\hline
\parbox{12cm}{
  \vspace{0.2cm}
  \textbf{12) Hvordan får jeg min yndlingschokolade i automaten?} \\
  \texttt{0}: Skriver den på brugerønskesedlen. \\
  \texttt{X}: Køber den i Netto og smider den hårdt ind ad lemmen. \\
  \texttt{1}: Ringer til vores forældre.
  \vspace{0.1cm}
} & & & \\\hline
\parbox{12cm}{
  \vspace{0.2cm}
  \textbf{13) Hvad gør man når der ikke er mere kaffe tilbage?}\\
  \texttt{0}:  Brygger ny kaffe \quad
  \texttt{X}:  Brygger ny kaffe \quad
  \texttt{1}:  Brygger ny kaffe
  \vspace{0.1cm}
} & & & \\\hline
\end{tabular}

\vspace{0.5cm}

{\textbf{BONUS FRITEKST-SPØRGSMÅL!!!} Du har travlt og når ikke at spise
  morgenmad hjemmefra. Du skynder dig derfor at lave en toast, og tager banan og
  en yoghurt med müsli. På vej ned til forelæsning kommer du i tanke om at du
  ikke har betalt. Hvor meget MobilePay'er du til kantinen?
\\ \vspace{0.5cm} \rule{5cm}{0.4pt}}

\vspace{0.5cm}
{\large Navn: \hrulefill}
\end{center}

\end{document}
